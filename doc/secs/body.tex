\section{Project Overview}
For our project, we implemented an interpreter for the dependently typed language \texttt{slyce}\footnotemark in Haskell.
\texttt{slyce} is a pure language that features $\lambda$ abstractions, let-expressions, if-then-else expressions, abstract data types, and pattern matching.
In addition to user defined abstract data types, the language  comes with several built in data types: namely \texttt{Unit}, \texttt{Bool}, and \texttt{Pair}.
\texttt{slyce} is also equipped with a parser and exhaustive error checking for program debugging.  
To implement \texttt{slyce}, we referenced tutorials for several dependently typed languages~\cite{friedman2018little, loh2010tutorial}, most notably Stephanie Weirich's \texttt{pi-forall} tutorial~\cite{weirich2022implementing}.

We came into this project knowing absolutely nothing about dependent types, and we have certainly learned a lot throughout this journey.
While some mysteries of dependent type checking still remain (covered in Section~\ref{reflection}), we have come away from this project with a deep understanding of much of Weirich's implementation and a core intuition about her decisions.
\footnotetext{\texttt{slyce} is a reference to the phrase "slice of pie", because pie sounds like $\Pi$, as in the $\Pi$-calculus, which is the language of dependent types. However, since our project is a simple toy implementation, it is a "slice of $\Pi$". Moreover, the "y" in the name looks like an upside-down $\lambda$, evoking how our language differs from the $\lambda$-calculus.}

\section{Installation}
 To run the type checker, first compile \texttt{slyce} using \texttt{stack build}. 
 Then, to run \texttt{slyce} on a source file, such as \texttt{hello.sly} located in the \texttt{/examples} subdirectory, execute the command\newline\centerline{\texttt{stack exec slyce -- -s -p -t ./examples/hello.sly}}\newline
 \texttt{-s} invokes the scanner for \texttt{slyce}, \texttt{-p} invokes the parser, and \texttt{-t} invokes the type checker. 
 If the file type checks, \texttt{slyce} will print the types of the variables and functions from the file to the terminal.
 Otherwise, the type checker will print an error message. 
\section{\texttt{slyce} example programs}
To give a taste for \texttt{slyce}, we first walk through a set of example programs that can be found in the \texttt{/examples} subdirectory of the \texttt{slyce} source code.
These examples provide a concrete reference for \texttt{slyce}'s key features discussed throughout the rest of this report.
\subsection{\texttt{vec.sly}: The \texttt{slyce} Hello World program}
\begin{figure}[h!]
\begin{lstlisting}
    data Nat where
        Zero,
        Succ of (Nat).

    zero = Zero.
    one = Succ zero.
    two = Succ one.
    three = Succ two.

    data Vec (a:U) (n:Nat) where
        Nil of (n = Zero),
        Cons of (m:Nat) (a) (Vec a m) (n = Succ m).

    head : (a:U) -> (n:Nat) -> Vec a (Succ n) -> a.
    head = \a. \n. \v.
        match v with
            | Cons m x xs -> x.

    v : Vec Nat three.
    v = Cons two one (Cons one two (Cons zero three Nil)).
\end{lstlisting}
\end{figure}

\subsection{\texttt{largeelim.sly}: Dependent pattern matching/propositional equality}
\begin{figure}[h!]
\begin{lstlisting}
    not : Bool -> Bool .
    not = \x. if x then False else True.
    
    t : Bool -> U.
    t = \b. if b then Unit else Bool.
    
    bar : (y : Bool) -> t y.
    bar = \b. if b then () else True.    
\end{lstlisting}
\end{figure}
\subsection{\texttt{sym.sly}: Definitional equality}
\begin{figure}[h!]
\begin{lstlisting}
    sym : (a:U) -> (x:a) -> (y:a) -> (x = y) -> y = x.
    sym = \a. \x. \y. \pf. subst Refl by pf.

    not : Bool -> Bool.
    not = \x. if x then False else True.

    false_is_not_true : False = (not True).
    false_is_not_true = Refl.

    not_true_is_false = sym Bool False (not True) false_is_not_true.
\end{lstlisting}
\end{figure}

\section{Our approach}
Our implementation process for \texttt{slyce} was iterative, and consisted of four main components when tackling any given feature.
\begin{enumerate}
    \item Read \texttt{pi-forall}'s documentation on the specific feature we sought to implement
    \item Attempt to implement the feature while referencing the documentation
    \item Check our approach with Weirich's source code
    \item Test our additions and debug (alot)
\end{enumerate}

Following the style of the \texttt{pi-forall} tutorial, we began with a simple, core language without any bells and whistles.
Our first iterations of \texttt{slyce} were without error messages or a parser.
Thus, step \textbf{5} of the implementation process was primitive in the at first. 
We began with simply loading our modules into Haskell's \texttt{ghci} environment and testing from there.
As we progressed farther in the tutorial and type checking features built in complexity, we added these tools.
This process ensured that we deeply understood the foundation of our type checker before we began to add new, more involved features. 

The core theories behind dependent type checking in \texttt{slyce} largely mirror that of \texttt{pi-forall}.
Weirich's tutorial covers the theory and leaves all the other features that must come with a dependent type checker up to interpretation.
For that reason, we diverge the most from \texttt{pi-forall} in our parser and error checking approaches, though upon reflection in Section~\ref{reflection}, we wish we had implemented our parser in the \texttt{pi-forall} style from the beginning.

\section{Background: Dependent types}
The key feature showcased in \texttt{slyce} is dependent type checking.

Dependent types are one direction in Barendregt's lambda cube describing types
that depend on terms. In a dependently typed language like \texttt{slyce},
little or no distinction is made between terms and types. Types may be
parameterized not just by other types, but by terms as well. For instance, a
type like \texttt{Vec (a:U) (n:Nat)} is parameterized by a type and a term, and specific instances
of this type in a type signature may even call type constructors, data
constructors, or functions that return types, such as \texttt{Succ (n:Nat)}.

The examples above
provide a good overview of some of the key features of dependently typed
languages. Some of the especially cool features of dependent types, which
motivated us to explore this area, include type safety and a robust system for
propositions as types.

Implementing dependent types requires the programmer to pay attention to many
subtle details that are not as relevant in other type systems such as System F.
The fundamental difference is that because types may depend on terms,
deciding whether two types (or terms) are equivalent for the purposes of type
checking is highly non-trivial, and necessarily involves some amount of
compile-time evaluation. Following Weirich's tutorial, we have accomplished
this using weak head normal form reduction of terms/types.

Some other difficulties that may arise when implementing dependent types may
include: unification, distinguishing constructors and functions, and many more
possibilities to introduce parser ambiguities because the line between types
and terms breaks down.

The subtleties of dependent types imply various approaches to their
implementation. The implementation that follows has its own advantages and
limitations. Regrettably, since
we focused mainly on implementing Weirich's tutorial, we are unable to provide
a detailed comparison between this and other approaches, nor are we confident
to describe the theoretical underpinnings of our particular implementation and
a proof of its properties. This would require a knowledge of dependent type
theory that we simply do not possess. However,
we hope that our future studies will expose us to the depth of dependent type
theory. Implementing this language has allowed us to dip our toes in the water,
and we are excited to dive in.

\section{Bidirectional type system}
The \texttt{slyce} type system makes use of \emph{bidirectional typing}. 
A bidirectional type system splits up type rules into two categories of judgements: \emph{inference} judgements and \emph{checking} judgements.
The following rules serve as a sketch for \texttt{slyce}'s core bidirectional type system, though they notably do not include rules for checking data constructors, equality types, and case expressions.
To get a sense for these typing rules, please refer to~\cite{weirich2022implementing}.
\subsection{Type Inference: $\Gamma\vdash a \Rightarrow A$}
Type inference, $\Gamma\vdash a \Rightarrow A$\footnotemark, dictates that in the context $\Gamma$, we should infer that a term $a$ has type $A$.
The inference rules for \texttt{slyce} are outlined below.
In many rules, type inference depends on type checking.

%% slyce inference rules for bidirectional type system
\begin{figure}[h!]
    \[
        \begin{bprooftree}
            \AxiomC{$x : A \in \Gamma$}
            \RightLabel{I-var}
            \UnaryInfC{$\Gamma\vdash x\Rightarrow A $}
        \end{bprooftree}\qquad
        \begin{bprooftree}
            \AxiomC{$ $}
            \RightLabel{I-type}
            \UnaryInfC{$\Gamma\vdash \mathbf{U}\Rightarrow \mathbf{U}$}
        \end{bprooftree}\qquad
        \begin{bprooftree}
            \AxiomC{\stackanchor{$\Gamma\vdash A \Leftarrow \mathbf{U}$}{$\Gamma, x : A\vdash B \Leftarrow \mathbf{U}$}}
            \RightLabel{I-Pi}
            \UnaryInfC{$\Gamma\vdash (x : A) \rightarrow B \Rightarrow \mathbf{U}$}
        \end{bprooftree}
    \]\newline
    \[
        \begin{bprooftree}
            \AxiomC{\stackanchor{$\Gamma\vdash A \Leftarrow \mathbf{U}$}{$\Gamma \vdash a \Leftarrow A$}}
            \RightLabel{I-ann}
            \UnaryInfC{$\Gamma\vdash (a : A) \Rightarrow A$}
        \end{bprooftree}\qquad
        \begin{bprooftree}
            \AxiomC{$\Gamma\vdash a \Rightarrow A$}
            \AxiomC{$\Gamma, x:A, x = a \vdash b \Rightarrow B$}
            \RightLabel{I-let}
            \BinaryInfC{$\Gamma\vdash \:\mathbf{let}\: x = a \:\mathbf{in}\: b \Rightarrow B[a/x]$}
        \end{bprooftree}
    \]\newline
    \[
        \begin{bprooftree}
            \AxiomC{$\Gamma\vdash a \Rightarrow A$}
            \AxiomC{$\mathbf{whnf}\: A \rightsquigarrow (x: A_{1}) \rightarrow B$}
            \AxiomC{$\Gamma\vdash b \Leftarrow A_{1}$}
            \RightLabel{I-app}
            \TrinaryInfC{$\Gamma\vdash a\: b \Rightarrow B[b/x]$}
        \end{bprooftree}
    \]
\end{figure}

Judgements I-app and I-let make use of \emph{definitional equality} to perform type inference. 
The explanation and implementation of this property can be found in section~\ref{equal}.

\subsection{Type Checking: $\Gamma\vdash a \Leftarrow A$}
Type checking, $\Gamma\vdash a \Leftarrow A$, makes use of information from the context $\Gamma$, like the types of top-level definitions, to look up the type of $a$ and check that it matches the known type $A$.
The checking rules for \texttt{slyce} are outlined below.
%% slyce inference rules for bidirectional type system
\begin{figure}[h!]
    \[
        \begin{bprooftree}
            \AxiomC{$\Gamma, x:A\vdash a \Leftarrow B$}
            \RightLabel{C-lambda}
            \UnaryInfC{$\Gamma \vdash \lambda x . a \Leftarrow (x:A) \rightarrow B$}
        \end{bprooftree}\qquad
        \begin{bprooftree}
            \AxiomC{$\Gamma\vdash a \Rightarrow A$}
            \AxiomC{$\Gamma, x: A, x = a \vdash b \Leftarrow B$}
            \RightLabel{C-let}
            \BinaryInfC{$\Gamma \vdash \mathbf{let}\: x = a\: \mathbf{in}\:b \Leftarrow B$}
        \end{bprooftree}
    \]\newline
    \[
        \begin{bprooftree}
            \AxiomC{\stackanchor{$\Gamma\vdash x \Leftarrow \mathbf{Bool}$}{
                {$\Gamma, x = \mathbf{True}\vdash b_{1} \Leftarrow A$}
                {$\:\:\:\:\:\:\:\Gamma, x = \mathbf{False}\vdash b_{2} \Leftarrow A$}
            }}
            \RightLabel{C-if}
            \UnaryInfC{$\Gamma \vdash \mathbf{if}\: x\: \mathbf{then}\: b_{1}\: \mathbf{else} \: b_{2} \Leftarrow A $}
        \end{bprooftree}
    \]\newline
    \[
        \begin{bprooftree}
            \AxiomC{$\Gamma\vdash z \Rightarrow (x : A_{1}\: * \:A_{2})$}
            \AxiomC{$\Gamma, x: B_{1}, y:B_{2}, z = (x, y)\vdash b \Leftarrow B[(x, y)/z]$}
            \RightLabel{C-letpair}
            \BinaryInfC{$ \Gamma \vdash \mathbf{let}\: (x, y)\: = \: z \: \mathbf{in}\:b\Leftarrow B$}
        \end{bprooftree}
    \]
\end{figure}

Judgements C-let, C-letpair and C-if make use of \emph{definitional equality} and \emph{propositional equality} to type check. 
The explanation and implementation of these properties can be found in Sections~\ref{equal} and~\ref{flow}.
\footnotetext{We are following the style of Weirich's bidirectional type system: $\Rightarrow$ for type inference and $\Leftarrow$ for type checking. These can be swapped for $\uparrow$ and $\downarrow$ respectively when referring to the bidirectional style outline in the lecture notes.}



\section{Haskell Implementation}
%key points from all of the files
\subsection{Ast.hs}
\subsection{Context.hs}
\subsection{TypeCheck.hs}
\subsection{Equality.hs}\label{equal}

\section{Key features}
We would like to highlight and discuss a few key features with an eye towards
implementation.

\subsection{$\Pi$ types}
The central feature of any dependent type system is the $\Pi$ type. This is the
type of functions where the output type may depend on the input type. In order
to implement this, our \texttt{Pi} constructor in the AST takes a name and type for the
input type and a term for the output type, where the name is bound in the
output type.

To type check lambda abstractions, we must check the type of the body with the
bound variable of the abstraction added to the context with the input type of
the \texttt{Pi} type.

To type check function application, we must instantiate the \texttt{Pi} type of
the function with the value being applied.

\subsection{Equality types}\label{equal}
With propositions as types, equality types are a very interesting feature of
our language. Equality types merely represent that two . They are associated
with two forms: \texttt{Refl} (a value of an equality type) and \texttt{Subst a
b}, which allows us to use equality proofs to transform arbitrary terms into
desired equivalent terms.

When type checking \texttt{Refl}, we must verify that the types it claims are
equal are in fact definitionally equal.

When type checking \texttt{Subst}, we verify that that proof passed in has
equality type, and then we add new definitions to the context that allow the
type checker to make use of equality to produce the desired new output type.

\subsection{Flow sensitivity}\label{flow}
Flow sensitivity is an optimization for eliminators like \texttt{If} that,
instead of just simply type checking each branch of the \texttt{If}, makes use
of the contextual information implied by the flow. Specifically, in \texttt{If}, when type checking the
consequent branch, if the condition can be reduced to a variable, we add a new
declaration to the context just for this branch that equates the condition with
\texttt{True} (and respectively for the \texttt{False} case). This can help
simplify the type checking of each branch.

We perform a similar optimization for the \texttt{LetPair} eliminator.

\subsection{Data types}
Data types are the feature that allows us to express more interesting
constructs, proofs, and programs in our language.

The implementation of data types is briefly described above, but it is too
complex to go into detail here. Please see our implementation, which contains
many comments documenting the process of type checking data types.

In order to effectively use data types, we implement general dependent pattern
matching via a \texttt{Match} expression, equivalent to Haskell's
\texttt{case x of ...} expression. The implementation of this involves
unifying each pattern with the reduced scrutinee in order to effectively
type check each body of the match expression.

\section{Discussion and reflections}\label{reflection}
Upon much reflection and experience with this tutorial, our main takeaway from this project is to \emph{always trust Stephanie Weirich}!
Many times throughout our development process, after checking our implementation of a feature with \texttt{pi-forall}, we would conclude that the subtle differences in our approaches were irrelevant.
Later down the line--sometimes much later--we would come to realize that those differences actually make or break the type checker.

A concrete example of this is the API call to the\newline \texttt{Unbound.Generics.LocallyNameless} library's freshness monad to generate fresh names for our terms. 
In \texttt{pi-forall} and \texttt{slyce}, the \texttt{Unbound.unbind} is used to unbind term names from their bodies in $\lambda$ expressions and $\Pi$ types.
When unbinding two different term names from their bodies and checking for propositional equivalence, \texttt{pi-forall} uses \texttt{Unbound.unbind2Plus} to unbind the two terms at the same time. 
After reading the documentation that was referred to in the tutorial on this call, we concluded that our original approach of unbinding the names separately was satisfactory. 
It was only after implementing extensive error messages, painstakingly tracing a program through the entire type checker, and hours of debugging when we realized that this decision was the source of very subtle naming error.
\texttt{Unbound.unbind2Plus} unbinds two names and gives them the same fresh name, while \texttt{Unbound.unbind}ing separately results in different names, eventually causing an error when performing equivalence checking.

%% insert stuff about parser here?
We suffered a similar comeuppance when, after initially deciding that it would
be simpler to use the Happy LALR parser generator instead of using an LL parser
combinator like Weirich, we found that our parser was unable to satisfactorily
parse data types. This was the last chapter and the last thing we implemented,
so it became a frantic search to find a way to distinguish constructors from
function application and type constructors from data constructors. We settled
on a solution that works for most cases, but due to time limitations, we are
left with numerous shift/reduce and reduce/reduce errors that make our parser
fragile; it frequently incorrectly parses function application. Had we more
time, we would either reimplement our parser using parser combinators like Weirich
does, or adopt the approach that SSLANG takes of using SYB's
\texttt{everywhere} function to modify the abstract syntax tree after an
initial parse.

One drawback of this reflection is that \texttt{pi-forall} is a delicate piece of software. 
It left very little room for experimentation or divergence from the source code, especially as we got further and further into the tutorial.
This is not necessarily a criticism of the tutorial, as it certainly
accomplishes its pedagogical goal and we learned a ton. This is more a
reflection on the complexity and subtlety of dependent type theory. That being
said, the difficulties we encountered forced us to really understand what was
going on. If Weirich had included more documentation or explanation of some
more arcane choices, we might not have come away with such a deep understanding
of our own implementation.

The specificity of the \texttt{pi-forall} language and its quirks also rendered
the many other dependent type checker tutorials unhelpful because little seemed
to carry over to Weirich's approach.
Even though many of the problems other tutorials tackle are the same, the approach tends to
be completely different, and in many cases, we found ourselves unsure about whether small
changes to the implementation would change important properties of the type
system, such as making it unsound. We lack confidence in our ability to
reason about how minor alterations to the type checker would have butterfly
effects on the properties of our system.

For example, we struggled with the lack of documentation and relevance of the \texttt{Unbound.instantiate} call for substitution when performing reduction via weak head normal form.
In our implementation, when performing substitution into the bodies of let-expressions and applications, we were first unbinding the binder and then performing the substitution. 
Weirich, on the other hand, uses the special \texttt{Unbound.instantiate} call which essentially combines these processes. 
Upon investigation, we discovered that Weirich herself implemented this feature into the library specifically for \texttt{pi-forall}, and \texttt{Unbound.instantiate} was just made available in the latest release of the library, though this tutorial was published last year.

Another example: early on, we figured there was no good reason that Weirich
processes functions and signatures one at a time, as opposed to adding them all
to the context at once. For simplicity, and because Weirich did not explain
this choice in the tutorial or the code documentation, we opted for the latter.
This ended up making our type system unsound, and a long session with ChatGPT
helped us understand how we could detect unsoundness through examples. We ended
up implementing an approach nearly identical to that of Weirich.

We believe that the process of "misconception to realization to correction" is a very valuable outcome of the project. 
The only way we could truly understand the value and inner workings of the implementation was to try it ourselves and then come to the conclusion that we were wrong, after re-examination.
We would not have been able to conceptualize much of \texttt{pi-forall}'s source code without this time-consuming yet ultimately rewarding process.






