\section{Project Overview}
For our project, we implemented an interpreter for the dependently typed language \texttt{slyce}\footnotemark in Haskell.
\texttt{slyce} is a pure language that features $\lambda$ abstractions, let-expressions, and if-then-else expressions.
To implement \texttt{slyce}, we referenced tutorials for several dependently typed languages~\cite{friedman2018little, loh2010tutorial}, most notably Stephanie Weirich's \texttt{pi-forall} tutorial~\cite{weirich2022implementing}.

\footnotetext{\texttt{slyce} is a reference to the phrase "slice of pie", because pie sounds like $\Pi$, as in the $\Pi$-calculus, which is the language of dependent types. However, since our project is a simple toy implementation, it is a "slice of $\Pi$". Moreover, the "y" in the name looks like an upside-down $\lambda$, evoking how our language differs from the $\lambda$-calculus.}
