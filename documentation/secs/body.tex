\section{Project Overview}
For our project, we implemented an interpreter for the dependently typed language \texttt{slyce}\footnotemark in Haskell.
\texttt{slyce} is a pure language that features $\lambda$ abstractions, let-expressions, and if-then-else expressions.
To implement \texttt{slyce}, we referenced tutorials for several dependently typed languages~\cite{friedman2018little, loh2010tutorial}, most notably Stephanie Weirich's \texttt{pi-forall} tutorial~\cite{weirich2022implementing}.

\footnotetext{\texttt{slyce} is a reference to the phrase "slice of pie", because pie sounds like $\Pi$, as in the $\Pi$-calculus, which is the language of dependent types. However, since our project is a simple toy implementation, it is a "slice of $\Pi$". Moreover, the "y" in the name looks like an upside-down $\lambda$, evoking how our language differs from the $\lambda$-calculus.}

\section{Installation}
To run the type checker, first compile \texttt{slyce} using \texttt{stack build}. 
Then, to run \texttt{slyce} on a source file, such as \texttt{hello.sly} located in the \texttt{/examples} subdirectory, execute the command\newline\centerline{\texttt{stack exec slyce -- -s -p -t ./examples/hello.sly}}\newline
\texttt{-s} invokes the scanner for \texttt{slyce}, \texttt{-p} invokes the parser, and \texttt{-t} invokes the type checker. 
If the file type checks, \texttt{slyce} will print the types of the variables and functions from the file to the terminal.
Otherwise, the type checker will print an error message. 

\section{Dependent Types}
The key feature showcased in \texttt{slyce} is dependent type checking. 
Dependent types, as 

\section{Language Tutorial}
\subsection{Hello World}
\section{Syntax and Semantics}
\section{Type System}
The \texttt{slyce} type system makes use of \emph{bidirectional typing}. 
A bidirectional type system splits up type rules into two categories of judgements: \emph{inference} judgements and \emph{checking} judgements.
Type inference, $\Gamma\vdash a \Rightarrow A$\footnotemark, dictates that in the context $\Gamma$, we should infer that a term $a$ has type $A$.
The inference rules for \texttt{slyce} are outlined below.
In many rules, type inference depends on type checking.
%% slyce inference rules for bidirectional type system
\begin{figure}[h!]
    \caption{Type inference rules for \texttt{slyce}}
    \[
        \begin{bprooftree}
            \AxiomC{$x : A \in \Gamma$}
            \RightLabel{I-var}
            \UnaryInfC{$\Gamma\vdash x\Rightarrow A $}
        \end{bprooftree}\qquad
        \begin{bprooftree}
            \AxiomC{$ $}
            \RightLabel{I-type}
            \UnaryInfC{$\Gamma\vdash \mathbf{U}\Rightarrow \mathbf{U}$}
        \end{bprooftree}\qquad
        \begin{bprooftree}
            \AxiomC{\stackanchor{$\Gamma\vdash A \Leftarrow \mathbf{U}$}{$\Gamma, x : A\vdash B \Leftarrow \mathbf{U}$}}
            \RightLabel{I-Pi}
            \UnaryInfC{$\Gamma\vdash (x : A) \rightarrow B \Rightarrow \mathbf{U}$}
        \end{bprooftree}
    \]\newline
    \[
        \begin{bprooftree}
            \AxiomC{\stackanchor{$\Gamma\vdash A \Leftarrow \mathbf{U}$}{$\Gamma \vdash a \Leftarrow A$}}
            \RightLabel{I-ann}
            \UnaryInfC{$\Gamma\vdash (a : A) \Rightarrow A$}
        \end{bprooftree}\qquad
        \begin{bprooftree}
            \AxiomC{$\Gamma\vdash a \Rightarrow A$}
            \AxiomC{$\Gamma, x:A, x = a \vdash b \Rightarrow B$}
            \RightLabel{I-let}
            \BinaryInfC{$\Gamma\vdash \:\mathbf{let}\: x = a \:\mathbf{in}\: b \Rightarrow B[a/x]$}
        \end{bprooftree}
    \]\newline
    \[
        \begin{bprooftree}
            \AxiomC{$\Gamma\vdash a \Rightarrow A$}
            \AxiomC{$\mathbf{whnf}\: A \rightsquigarrow (x: A_{1}) \rightarrow B$}
            \AxiomC{$\Gamma\vdash b \Leftarrow A_{1}$}
            \RightLabel{I-app}
            \TrinaryInfC{$\Gamma\vdash a\: b \Rightarrow B[b/x]$}
        \end{bprooftree}
    \]
\end{figure}
\footnotetext{We are following the style of Weirich's bidirectional type system: $\Rightarrow$ for type inference and $\Leftarrow$ for type checking. These can be swapped for $\uparrow$ and $\downarrow$ respectively when reffering to the bidirectional style outline in the lecture notes.}



\section{Haskell Implementation}
%key points from all of the files
\subsection{Ast.hs}
\subsection{Context.hs}
\subsection{TypeCheck.hs}
\subsection{Equality.hs}
\section{Discussion (questions, comments, lamentations, etc.)}

