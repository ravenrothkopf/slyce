\section{Our approach}
Our implementation process for \texttt{slyce} was iterative, and consisted of four main components when tackling any given feature.
\begin{enumerate}
    \item Read \texttt{pi-forall}'s explanation of the specific feature we sought to implement
    \item Attempt to implement the feature while referencing the explanation
    \item Compare our approach with Weirich's source code
    \item Test our additions with some examples and debug (alot)
\end{enumerate}

Following the style of the \texttt{pi-forall} tutorial, we began with a simple, core language without any bells and whistles.
Our first iterations of \texttt{slyce} were without error messages or a parser.
Thus, step \textbf{5} of the implementation process was originally primitive. 
We began with simply loading our modules into Haskell's \texttt{ghci} environment and testing from there.
As we progressed farther in the tutorial and type checking features built in
complexity, we added these tools in response to the need for convenience of
testing and
better error reporting for debugging.
This process ensured that we deeply understood the foundation of our type checker before we began to add new, more involved features. 
It also allowed us to motivate our own development of the peripheral features
of our language; since Weirich's tutorial focuses on only the type checker and
a few other components, we had the freedom to adopt our own approach to the
parser, pretty printer, error tracer, and datatypes.

The core theories behind dependent type checking in \texttt{slyce} largely mirror that of \texttt{pi-forall}.
Weirich's tutorial covers the theory and leaves all the other features that
must come with a usable dependent type checker up to interpretation.
For that reason, we diverge the most from \texttt{pi-forall} in our parser and
error tracing approaches, though upon reflection in Section~\ref{reflection},
we wish we had implemented our parser in the \texttt{pi-forall} style from the
beginning due to fundamental differences between parser combinators and parser
generators.
