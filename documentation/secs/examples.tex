\section{\texttt{slyce} example programs}
To give a taste for \texttt{slyce}, we first walk through a set of example
programs that can be found in the \texttt{examples/} subdirectory of the \texttt{slyce} source code.
These examples provide a concrete reference for \texttt{slyce}'s key features discussed throughout the rest of this report.
While we do not explicitly cover the syntax of \texttt{slyce}, we hope that a
pass over these examples, as well as the rest in the \texttt{examples/} subdirectory, serve as an extensive set of use cases.
The full syntax is available in the \texttt{slyce} parser in \texttt{Parser.hs}.
 
\subsection{\texttt{vec.sly}: The "Hello World" of dependent types}
\begin{figure}[h!]
\begin{lstlisting}
    data Nat where
        Zero,
        Succ of (Nat).

    zero = Zero.
    one = Succ zero.
    two = Succ one.
    three = Succ two.

    data Vec (a:U) (n:Nat) where
        Nil of (n = Zero),
        Cons of (m:Nat) (a) (Vec a m) (n = Succ m).

    head : (a:U) -> (n:Nat) -> Vec a (Succ n) -> a.
    head = \a. \n. \v.
        match v with
            | Cons m x xs -> x.

    v : Vec Nat three.
    v = Cons two one (Cons one two (Cons zero three Nil)).
\end{lstlisting}
\end{figure}

The classic use case for dependent types is a vector parameterized by the type
of its elements and its length.
In this program, we showcase the use of data types, $\Pi$ types, function declaration, and pattern matching.

The type-safe vector data type, \texttt{Vec} is indexed over a specific length using the dependent type system. 
The implementation defines two new data types: \texttt{Nat} and \texttt{Vec}. 
\texttt{Nat} represents natural numbers and is defined recursively as either Zero or the successor of another Nat. 
\texttt{Vec} represents a vector of values of type \texttt{a}, which has type \texttt{U} -- \texttt{slyce}'s type of types -- that has a length of \texttt{n} elements.

The \texttt{Vec} data type has two constructors: \texttt{Nil} and \texttt{Cons}. 
\texttt{Nil} creates an empty vector with a length of zero, and \texttt{Cons}
adds an element of type \texttt{a} to the beginning of an existing vector of
length \texttt{m}, resulting in a new vector with the type same type and a
length \texttt{Succ m}, i.e. the length of the tail incremented by one. 
The type of the \texttt{Cons} constructor includes a proposition that the length of
the resulting vector is one more than the length of the input vector. This is
not an argument to the constructor, but constraint on the other arguments.

The \texttt{head} function takes a \texttt{Vec} of length \texttt{Succ n} and
returns its first element, which has type \texttt{a}. 
The function pattern matches on the input vector, using the \texttt{Cons} constructor to extract the head element.
Since the vector has length \texttt{Succ n}, it cannot be empty, so this
function is type safe: it cannot result in a runtime error, since passing it an
empty vector is a compile-time type error.

As an example of how to use the vector data type constructors, we include a
definition of a vector of type \texttt{Vec Nat three} (i.e., a vector of
three natural numbers). In each invocation of \texttt{Cons}, the first argument
is the length of the tail vector and the second argument is the element to
prepend to the front of the vector.

\subsection{\texttt{list.sly}: Lists}
\begin{figure}[h!]
\begin{lstlisting}
    data List (t:U) where
        Nil,
        Cons of (x:t) (xs:List t).

    map : (a:U) -> (b:U) -> (a -> b) -> List a -> List b.
    map = \a. \b. \f. \l.
        match l with
            | Nil -> Nil
            | Cons x xs -> Cons (f x) (map a b f xs).

    foldr : (a:U) -> (b:U) -> (a -> b -> b) -> b -> List a -> b.
    foldr = \a. \b. \f. \acc. \l.
        match l with
          | Nil -> acc
          | Cons x xs -> f x (foldr a b f acc xs).

    any : List Bool -> Bool.
    any = foldr Bool Bool or False.

    all : List Bool -> Bool.
    all = foldr Bool Bool and True.
\end{lstlisting}
\end{figure}
This example demonstrates a small library of list functions. Lists are written
as in Haskell: they are parameterized by the type of their elements, but do not
include information about their length.

Due to parametric polymorphism, the polymorphic \texttt{map} and \texttt{fold}
functions must take type arguments.

\subsection{\texttt{largeelim.sly}: Propositional equality}
\begin{figure}[h!]
\begin{lstlisting}
    not : Bool -> Bool.
    not = \x. if x then False else True.
    
    t : Bool -> U.
    t = \b. if b then Unit else Bool.
    
    bar : (y : Bool) -> t y.
    bar = \b. if b then () else True.    

    x : Unit.
    x = bar True.

    y : Bool.
    y = bar False.

    z : (Unit = t True).
    z = Refl.

    w : (Bool = t False).
    w = Refl.
\end{lstlisting}
\end{figure}
This example demonstrates a simple use of dependent types to write an use a
function, \texttt{bar}, whose output type depends on the value of its input.

The use of \texttt{bar} is shown in \texttt{x} and \texttt{y}, which have
different types depending on what value was passed to \texttt{bar}.

Furthermore, we use propositional equality to demonstrate that \texttt{t True}
really is equal to \texttt{Unit}, and respectively for \texttt{False} and
\texttt{Bool}.

\texttt{Refl} is a language built-in proof of reflexivity, i.e., witness of an equality type that
corresponds to a proposition that two propositions are equal.
Since these equality propositions are true by definition of \texttt{t}, the
program type checks. This works due to our implementation of definitional
equality, which reduces terms to weak head normal form when checking if they
are equivalent. See implementation for more details.

If we swapped the arguments to the types of \texttt{z} and \texttt{w}, the
program would fail to type check, as it cannot prove that \texttt{t False} is
equal to \texttt{Unit}... because it isn't!

\subsection{\texttt{sym.sly}: Propositional equality}
\begin{figure}[h!]
\begin{lstlisting}
    sym : (a:U) -> (x:a) -> (y:a) -> (x = y) -> y = x.
    sym = \a. \x. \y. \pf. subst Refl by pf.

    not : Bool -> Bool.
    not = \x. if x then False else True.

    false_is_not_true : False = (not True).
    false_is_not_true = Refl.

    not_true_is_false = sym Bool False (not True) false_is_not_true.
\end{lstlisting}
\end{figure}
To elaborate further on propositional equality, we exhibit a function that
proves the symmetric property of equality.

Given a proof, \texttt{pf}, that proposition \texttt{x} equals proposition \texttt{y}
, \texttt{sym} returns a proof that \texttt{y = x}.
It achieves this by using the built-in function \texttt{subst}, which in this
case substitutes the reflexivity proof into the proof of equality, thus
swapping the variables and returning a new proof.

The rest of the file is an example of using this symmetry prover to show that
\texttt{False = not True} implies \texttt{not True = False}. This would fail to
type check if we gave it an untrue equality proposition.

Propositional equality is a deep and subtle art which we have not fully
explored or understood. This aspect of \texttt{slyce} connects us closely to
dependently typed languages like Agda and Coq, which prominently feature the
application of dependent types to logic and theorem proving.
