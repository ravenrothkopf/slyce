\section{\texttt{slyce} example programs}
To give a taste for \texttt{slyce}, we first walk through a set of example programs that can be found in the \texttt{/examples} subdirectory of the \texttt{slyce} source code.
These examples provide a concrete reference for \texttt{slyce}'s key features discussed throughout the rest of this report.
\subsection{\texttt{vec.sly}: The \texttt{slyce} Hello World program}
\begin{figure}[h!]
\begin{lstlisting}
    data Nat where
        Zero,
        Succ of (Nat).

    zero = Zero.
    one = Succ zero.
    two = Succ one.
    three = Succ two.

    data Vec (a:U) (n:Nat) where
        Nil of (n = Zero),
        Cons of (m:Nat) (a) (Vec a m) (n = Succ m).

    head : (a:U) -> (n:Nat) -> Vec a (Succ n) -> a.
    head = \a. \n. \v.
        match v with
            | Cons m x xs -> x.

    v : Vec Nat three.
    v = Cons two one (Cons one two (Cons zero three Nil)).
\end{lstlisting}
\end{figure}

\subsection{\texttt{largeelim.sly}: Dependent pattern matching/propositional equality}
\begin{figure}[h!]
\begin{lstlisting}
    not : Bool -> Bool .
    not = \x. if x then False else True.
    
    t : Bool -> U.
    t = \b. if b then Unit else Bool.
    
    bar : (y : Bool) -> t y.
    bar = \b. if b then () else True.    
\end{lstlisting}
\end{figure}
\subsection{\texttt{sym.sly}: Definitional equality}
\begin{figure}[h!]
\begin{lstlisting}
    sym : (a:U) -> (x:a) -> (y:a) -> (x = y) -> y = x.
    sym = \a. \x. \y. \pf. subst Refl by pf.

    not : Bool -> Bool.
    not = \x. if x then False else True.

    false_is_not_true : False = (not True).
    false_is_not_true = Refl.

    not_true_is_false = sym Bool False (not True) false_is_not_true.
\end{lstlisting}
\end{figure}
