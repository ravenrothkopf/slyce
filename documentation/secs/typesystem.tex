\section{Bidirectional type system}
The \texttt{slyce} type system makes use of \emph{bidirectional typing}. 
A bidirectional type system splits up type rules into two categories of judgements: \emph{inference} judgements and \emph{checking} judgements.
\subsection{Type Inference: $\Gamma\vdash a \Rightarrow A$}
Type inference, $\Gamma\vdash a \Rightarrow A$\footnotemark, dictates that in the context $\Gamma$, we should infer that a term $a$ has type $A$.
The inference rules for \texttt{slyce} are outlined below.
In many rules, type inference depends on type checking.

%% slyce inference rules for bidirectional type system
\begin{figure}[h!]
    \[
        \begin{bprooftree}
            \AxiomC{$x : A \in \Gamma$}
            \RightLabel{I-var}
            \UnaryInfC{$\Gamma\vdash x\Rightarrow A $}
        \end{bprooftree}\qquad
        \begin{bprooftree}
            \AxiomC{$ $}
            \RightLabel{I-type}
            \UnaryInfC{$\Gamma\vdash \mathbf{U}\Rightarrow \mathbf{U}$}
        \end{bprooftree}\qquad
        \begin{bprooftree}
            \AxiomC{\stackanchor{$\Gamma\vdash A \Leftarrow \mathbf{U}$}{$\Gamma, x : A\vdash B \Leftarrow \mathbf{U}$}}
            \RightLabel{I-Pi}
            \UnaryInfC{$\Gamma\vdash (x : A) \rightarrow B \Rightarrow \mathbf{U}$}
        \end{bprooftree}
    \]\newline
    \[
        \begin{bprooftree}
            \AxiomC{\stackanchor{$\Gamma\vdash A \Leftarrow \mathbf{U}$}{$\Gamma \vdash a \Leftarrow A$}}
            \RightLabel{I-ann}
            \UnaryInfC{$\Gamma\vdash (a : A) \Rightarrow A$}
        \end{bprooftree}\qquad
        \begin{bprooftree}
            \AxiomC{$\Gamma\vdash a \Rightarrow A$}
            \AxiomC{$\Gamma, x:A, x = a \vdash b \Rightarrow B$}
            \RightLabel{I-let}
            \BinaryInfC{$\Gamma\vdash \:\mathbf{let}\: x = a \:\mathbf{in}\: b \Rightarrow B[a/x]$}
        \end{bprooftree}
    \]\newline
    \[
        \begin{bprooftree}
            \AxiomC{$\Gamma\vdash a \Rightarrow A$}
            \AxiomC{$\mathbf{whnf}\: A \rightsquigarrow (x: A_{1}) \rightarrow B$}
            \AxiomC{$\Gamma\vdash b \Leftarrow A_{1}$}
            \RightLabel{I-app}
            \TrinaryInfC{$\Gamma\vdash a\: b \Rightarrow B[b/x]$}
        \end{bprooftree}
    \]
\end{figure}

Judgements I-app and I-let make use of \emph{definitional equality} to type check. The explanation and implementation of this property can be found in section~\ref{equal}.

\subsection{Type Checking: $\Gamma\vdash a \Leftarrow A$}

%% slyce inference rules for bidirectional type system
\begin{figure}[h!]
    \[
        \begin{bprooftree}
            \AxiomC{$\Gamma, x:A\vdash a \Leftarrow B$}
            \RightLabel{C-lambda}
            \UnaryInfC{$\Gamma \vdash \lambda x . a \Leftarrow (x:A) \rightarrow B$}
        \end{bprooftree}\qquad
        \begin{bprooftree}
            \AxiomC{$\Gamma\vdash a \Rightarrow A$}
            \AxiomC{$\Gamma, x: A, x = a \vdash b \Leftarrow B$}
            \RightLabel{C-let}
            \BinaryInfC{$\Gamma \vdash \mathbf{let}\: x = a\: \mathbf{in}\:b \Leftarrow B$}
        \end{bprooftree}
    \]\newline
    \[
        \begin{bprooftree}
            \AxiomC{\stackanchor{$\Gamma\vdash x \Leftarrow \mathbf{Bool}$}{
                {$\Gamma, x = \mathbf{True}\vdash b_{1} \Leftarrow A$}
                {$\:\:\:\:\:\:\:\Gamma, x = \mathbf{False}\vdash b_{2} \Leftarrow A$}
            }}
            \RightLabel{C-if}
            \UnaryInfC{$\Gamma \vdash \mathbf{if}\: x\: \mathbf{then}\: b_{1}\: \mathbf{else} \: b_{2} \Leftarrow A $}
        \end{bprooftree}
    \]\newline
    \[
        \begin{bprooftree}
            \AxiomC{$\Gamma\vdash z \Rightarrow (x : A_{1}\: * \:A_{2})$}
            \AxiomC{$\Gamma, x: B_{1}, y:B_{2}, z = (x, y)\vdash b \Leftarrow B[(x, y)/z]$}
            \RightLabel{C-letpair}
            \BinaryInfC{$ \Gamma \vdash \mathbf{let}\: (x, y)\: = \: z \: \mathbf{in}\:b\Leftarrow B$}
        \end{bprooftree}
    \]
\end{figure}

\footnotetext{We are following the style of Weirich's bidirectional type system: $\Rightarrow$ for type inference and $\Leftarrow$ for type checking. These can be swapped for $\uparrow$ and $\downarrow$ respectively when reffering to the bidirectional style outline in the lecture notes.}

